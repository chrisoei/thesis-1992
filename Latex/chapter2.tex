\typeout{}
\typeout{Compiling chapter2.tex}
\chapter{History}

The question of what kind of problem is difficult for a GA started
with Bledsoe~(1961).  Bledsoe described a situation he called
{\em lethal dependence} in which a mutation in each of two genes would 
be an improvement, but a mutation in either gene alone would lead to
death.  

Following Holland's schema theorem (Holland, 1975), Bethke~(1981)
introduced the Walsh-schema transform for computing schema averages
and gave some intuitive
conditions for problem difficulty that depended on the smoothness of the fitness
function and the asymptotic behavior of the Walsh coefficients.

The smoothness arguments were also used by
Weinberger, who started with Eigen's model for natural selection (Weinberger,
1987) and studied correlation lengths on the ``landscape'' of the fitness
function (Weinberger, 1988; Weinberger, 1990).
Kauffman in (Kauffman, 1989; Kauffman, 1990; Kauffman \& Levin, 1987)
 created and analyzed a model fitness landscape with tunable ruggedness, which
was analyzed further in Manderick, de Weger, \& Spiessens~(1991).
Lipsitch~(1991) used cellular automata rules to generate fitness
landscapes and performed simulations to discover which classes of cellular 
automaton created the hardest landscapes.  These analyses, however,
dealt with hill-climbing on fitness functions and neglected the effects
of crossover and the usefulness of schemata.  Also, Goldberg~(1990b)
pointed out that the asymptotic behavior of the Walsh coefficients
(and therefore the smoothness of the function) is not enough to
insure the growth of important schemata.

Holland's schema theorem came back into play when Goldberg~(1987)
defined deception and introduced the minimal deceptive problem.
The conditions for full static deception, a situation in which
all low-order schemata are misleading, was introduced
in Goldberg~(1989b).  The analysis of full static deception taking
into account the schema disruption due to crossover and mutation was
done in Goldberg~(1989c) using operator-adjusted Walsh coefficients.

This line of reasoning gave rise to a host of techniques:
a method for analyzing a GA population in which schemata are not
distributed uniformly (Bridges, Goldberg, 1989),
a method that uses Hadamard transforms to organize the deceptive conditions
(Homaifar \& Qi, 1990; Homaifar, Qi, \& Fost, 1991),
methods for constructing fully deceptive and intermediate deceptive
functions (Deb \& Goldberg, 1991; Goldberg, 1990b; Liepins \& Vose, 1991;
Whitley, 1991a), and a simpler set of criteria sufficient to
insure deception (Deb \& Goldberg, 1992).
This theory of deception has been used to explain some experimental
results, such as why a problem that is easy for a GA might be difficult
for a hill-climber (Wilson, 1991), and why certain GA test suites were
solved so easily (Das \& Whitley, 1991; Davis, 1991).
It also lead to some attacks 
(Forrest \& Mitchell, 1991; Grefenstette, 1991; Mitchell \& Forrest, 1991; 
Mitchell, Forrest, \& Holland, 1991; Tanese, 1989) that the lack of deception,
where
deception is defined by misleading schemata averages, is not enough by itself
to insure GA convergence to the global optimum.

Another mode of GA failure, apart from deception, has to do with the
variance of schema fitnesses and sampling error
(Davidor, 1991; Liepins \& Vose, 1990b; Schaffer, Eschelman \& Offut, 1991).
There were a number of studies that introduced techniques
for calculating schema fitness variances and 
signal-to-noise ratios and explain how to size a GA population accordingly
(Goldberg, Deb, \& Clark, 1991; Goldberg \& Rudnick, 1991;
Rudnick, 1991; Rudnick \& Goldberg 1991).
Connected with the issue of variance is multimodality, or having many
high peaks that may confuse the GA (Goldberg, Deb \& Horn, 1992).
Goldberg, Deb, \& Clark~(1991) give an overview and a summary of
conditions for GA success.

Generalizations of this basic theory of convergence include applications
to permutation problems 
(Kargupta, Deb, \& Goldberg, 1992; Sikora, 1991)
and fitness functions with real parameters (Goldberg, 1990a;
Wright, 1991).
On a more abstract level are generalizations of the schema notion to arbitrary
linear combinations of bits (Liepins \& Vose, 1990a), and arbitrary
predicates (Vose \& Liepins, 1991; Radcliffe, 1991).
Mason~(1991) generalized the notions of partition coefficients and
static deception to finite non-binary alphabets.

The other approaches to the question of GA convergence are few and
sparse.
Hart and Belew~(1991) points out that the general problem that the GA
tries to solve is NP-hard.
Kauffman~(1990) uses some concepts from information theory and
the physics of phase transitions to show a connection between
information redundancy and evolvability: minimal systems are not
evolvable due to lethally dependent parameters.

The work done on binary-coded GAs provides an extensive theoretical
framework that hinges on the Walsh-schema transform.
This thesis will focus on the theory of static deception generalized
to non-binary alphabets and begins by generalizing the Walsh transform.


\typeout{}
\typeout{Compiling abstract.tex}

\begin{abstractpage}
The Walsh transform is used extensively as a tool in determining whether a
fitness function over a binary string is deceptive or not.  This thesis shows
that the Walsh transform method for detecting deception is easily generalized to
functions over non-binary strings such as ternary strings, strings with
real parameters, and strings with some binary and ternary characters and
some real parameters.
  A generalization of the Hadamard transform
is then used to organize the generalized Walsh coefficients into
conditions for static deception for non-binary alphabets.
The variances of fitness of schemata are calculated using generalized
Walsh coefficients.  Mathematica code for performing most of the 
calculations mentioned is included.
\end{abstractpage}


\typeout{}
\typeout{Compiling appendix2.tex}
\chapter{Another Basis for Generalized Transforms}

As in Fourier series, we can use sines and cosines as well as complex
exponentials as our basis functions.  The disadvantage of using sines
and cosines as basis functions is that the basic theorems become somewhat
more complex.  However, the advantage is that all the Walsh coefficients
of a real function are real in this basis.  We will refer to the generalized
Walsh functions and transforms using sines as cosines as real generalized
Walsh functions and transforms.

Since $\cos(x)=\frac{e^{\imath x}+e^{-\imath x}}{2}$ and
$\sin(x)=\frac{e^{\imath x}-e^{-\imath x}}{2 \imath}$,
one can convert between the generalized Walsh coefficients used in the previous
sections to the real generalized Walsh coefficients.

To start, consider a function over strings of length 1.
There are two cases for the transform itself, one for even k and one for odd k.

{\noindent\bf Odd k}  The real generalized Walsh transform is given by
\begin{eqnarray}
a_0 &=& w_0, \nonumber \\
a_j &=& \frac{w_j + w_{k-j}}{2^{1/2}} \mbox{\ for $j \ne 0$},\nonumber \\
b_j &=& \frac{\imath}{2^{1/2}} (w_j - w_{k-j}).
        \label{ajbj1}
\end{eqnarray}
The inverse real generalized Walsh transform is given by the following:
\begin{equation}
f(x)=\frac{1}{k^{1/2}}(a_0+\sum_{j=1}^{\frac{k-1}{2}}{(a_j 2^{1/2} \cos(2 \pi j
x/k) +
         b_j 2^{1/2}\sin(2 \pi j x/k))}).
        \label{fofx1}
\end{equation}
Of course, one can express the $a_j$s and $b_j$s in terms of the inner products
of the cosines and sines with the function $f$:
\begin{eqnarray}
a_0 &=& \frac{1}{k^{1/2}} \sum_{x=0}^{k-1}{f(x)};\nonumber \\
a_j &=& \frac{2^{1/2}}{k^{1/2}} \sum_{x=0}^{k-1} {\cos(2 \pi j x/k) f(x)}
        \mbox{\ for $j\ne 0$};\nonumber \\
b_j &=& \frac{2^{1/2}}{k^{1/2}} \sum_{x=0}^{k-1} {\sin(2 \pi j x/k) f(x)}.
\end{eqnarray}
{\bf Even k}
\begin{eqnarray}
a_0 &=& w_0; \nonumber \\
a_{k/2} &=& 2^{-{1/2}} w_{k/2}; \nonumber \\
a_j &=& \frac{w_j + w_{k-j}}{2^{1/2}} \mbox{\ for $j\ne 0$ and $j\ne k/2$}; \nonumber \\
b_j &=& \frac{\imath}{2^{1/2}} (w_j - w_{k-j}).
                \label{ajbj}
\end{eqnarray}
\begin{eqnarray}
f(x)=\frac{1}{k^{1/2}}(a_0+\sum_{j=1}^{\frac{k}{2}-1}{(a_j 2^{1/2}\cos(2 \pi j x
/k)} &+&
        b_j 2^{1/2} \sin(2 \pi j x/k))\nonumber\\
        &+&2^{1/2} \cos(\pi x) a_{k/2}).
                \label{fofx}
\end{eqnarray}
Again, the coefficients can be expressed in terms of the inner product
of sines and cosines with $f(x)$:
\begin{eqnarray}
a_0 &=& \frac{1}{k^{1/2}} \sum_{x=0}^{k-1}{f(x)}; \nonumber \\
a_j &=& \frac{2^{1/2}}{k^{1/2}} \sum_{x=0}^{k-1} {\cos(2 \pi j x/k) f(x)}
        \mbox{\ for $j\ne 0$};\nonumber \\
b_j &=& \frac{2^{1/2}}{k^{1/2}} \sum_{x=0}^{k-1} {\sin(2 \pi j x/k) f(x)}.
\end{eqnarray}


The proof that the inverse real generalized Walsh transform of the
real generalized Walsh transform of a function is the function itself
comes from substituting the expressions
for $a_j$ and $b_j$~(\ref{ajbj1},\ref{ajbj}) into the expressions for
$f(x)$~(\ref{fofx1},\ref{fofx})
and verifying that it
indeed is the inverse generalized Walsh transform (\ref{inverse}).

There are two ways of generalizing the above method to $n$ dimensions.
The first is simply to take the Fourier transform along each dimension
as we did before, but to use sines and cosines as we have done above.
The problem with using this method is mainly notational complexity,
although the idea is just as simple
as the generalized Walsh transforms we discussed before;  all we are
doing is taking the Fourier transform in an $n$-dimensional space, using
sines and cosines.
As this first method of generalizing the sine and cosine transform becomes
cumbersome for long strings, it will not be pursued any further in this
thesis.

The second method of generalizing the Walsh
transform using sines and cosines works as follows:
\begin{eqnarray}
a_{\vec{0}} &=& \prod_m{k_m^{-1/2}} \sum_{\vec{x}} {f(\vec{x})}; \nonumber\\
a_{\vec{\jmath}} &=& 2^{1/2} \prod_m{k_m^{-1/2}} \sum_{x_1,x_2,\ldots,x_n}
        {\cos(2 \pi (j_1 x_1/k_1+j_2 x_2/k_2+\ldots j_n x_n/k_n)
        f(\vec{x})}
        \mbox{\ for $j\ne 0$};\nonumber \\
b_{\vec{\jmath}} &=& 2^{1/2} \prod_m{k^{-1/2}} \sum_{x_1,x_2,\ldots,x_n}
        {\cos(2 \pi (j_1 x_1/k_1+j_2 x_2/k_2+\ldots j_n x_n/k_n)
        f(\vec{x})}.
\end{eqnarray}
\begin{eqnarray}
f(\vec{x}) &=& \prod_m{k_m^{-1/2}} \sum_{\vec{\jmath}}
{}\{
a_{\vec 0}+
a_{\vec{\jmath}} 2^{1/2} \cos(2 \pi (j_1 x_1/k_1+j_2 x_2/k_2+\ldots j_n x_n/k_n)
)
        \nonumber\\
&& +
b_{\vec{\jmath}} 2^{1/2} \sin(2 \pi (j_1 x_1/k_1+j_2 x_2/k_2+\ldots j_n x_n/k_n)
)
\}.
\end{eqnarray}
This method has the disadvantage that the Walsh functions in $n$
dimensions is not just the product of Walsh functions in one dimension.
Explicitly, they are the following:
\begin{definition}[Real \veckary Walsh Functions]
\begin{eqnarray}
\Omega^{(\vec k)}_0(\vec x) &=& \prod_m{k_m^{-1/2}};\nonumber\\
\Omega^{(\vec k)}_{\vec{\jmath}}(\vec x) &=& 2^{1/2}\prod_m{k_m^{-1/2}} \cos(2 \
pi (j_1 x_1/k_1 +
        j_2 x_2/k_2 + \ldots j_n x_n /k_n)) \mbox{\ for $j\ne 0$};\nonumber\\
\Lambda^{(\vec k)}_{\vec{\jmath}}(\vec x) &=& 2^{1/2}\prod_m{k_m^{-1/2}} \sin(2
\pi (j_1 x_1/k_1 +
        j_2 x_2/k_2 + \ldots j_n x_n /k_n)).
\end{eqnarray}
\end{definition}


This transform also has the useful property that function averages over schema
with $m$ fixed positions become sums over $n-m$ positions in the transformed
space.
\begin{example}
Consider again a function $f$ over strings of
length 2 whose characters are taken from a ternary alphabet.  Some schemas
averages in terms of the transform coefficients are
\begin{eqnarray}
f(**) &=& \frac{1}{3} a_{(0,0)}; \nonumber \\
f(*0) &=& \frac{1}{3} (a_{(0,0)} + a_{(0,1)}); \nonumber\\
f(*1) &=& \frac{1}{3} (a_{(0,0)} -\frac{1}{2} a_{(0,1)} +\frac{3^{1/2}}{2} b_{(0
,1)}); \nonumber\\
f(*2) &=& \frac{1}{3} (a_{(0,0)} -\frac{1}{2} a_{(0,1)} -\frac{3^{1/2}}{2} b_{(0
,1)}).
\end{eqnarray}
\end{example}


\begin{theorem}
Consider a schema has $m$ fixed characters $p_i$ at positions $j_i$.
Then the average of $f$ over that schema is
\begin{eqnarray}
\prod_q{k_q^{-1/2}}
\sum_{l_1} \sum_{l_2} \ldots \sum_{l_m} & &
\cos(2 \pi (l_1 p_1/k_{j_1}+l_2 p_2/k_{j_2} +\ldots+l_m p_m/k_{j_m}))\nonumber\\
&& a_{(0,0,\ldots,0,l_1,0,\ldots,0,l_2,0,\ldots,0,l_m,0,\ldots)}\nonumber\\
&+&
\sin(2 \pi (l_1 p_1/k_{j_1}+l_2 p_2/k_{j_2} +\ldots+l_m p_m/k_{j_m}))\nonumber\\
&& b_{(0,0,\ldots,0,l_1,0,\ldots,0,l_2,0,\ldots,0,l_m,0,\ldots)}.
\end{eqnarray}
\end{theorem}



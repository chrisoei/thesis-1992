\typeout{}
\typeout{Compiling chapter4.tex}

\chapter{Using Generalized Walsh Coefficients to Determine Schema Averages}
Determining schema averages is at the heart for the reason of using these
transform methods.  
In Chapter~3, this thesis stated that a sum of $f$ over $m$ 
characters
of a string of length $n$ turned into a sum of generalized Walsh
coefficients over $n-m$ characters.  For this reason, the average of a function
over a schema with $m$ positions fixed turns into a sum over $m$ characters
of the Walsh coefficients.

\begin{example} Consider strings of length 3 taken from a (2,4,3)-ary alphabet.
Some examples of schema averages are as follows:
\begin{eqnarray}
f((*,*,*)) &=& \frac{1}{4} w_{(0,0,0)}; \nonumber \\
f((0,*,*)) &=& \frac{1}{4} (w_{(0,0,0)}+w_{(1,0,0)}); \nonumber \\
f((1,*,*)) &=& \frac{1}{4} (w_{(0,0,0)}-w_{(1,0,0)}); \nonumber \\
f((*,0,*)) &=& \frac{1}{4} (w_{(0,0,0)}+
				w_{(0,0,1)}+
				w_{(1,0,0)}+
				w_{(1,0,1)}); \nonumber \\
f((*,1,*)) &=& \frac{1}{4} (w_{(0,0,0)}+ \imath w_{(0,0,1)}-
				w_{(1,0,0)}-\imath w_{(1,0,1)}); \nonumber \\
f((*,2,*)) &=& \frac{1}{4} (w_{(0,0,0)}- w_{(0,0,1)}+
				w_{(1,0,0)}- w_{(1,0,1)}); \nonumber \\
f((*,3,*)) &=& \frac{1}{4} (w_{(0,0,0)}- \imath w_{(0,0,1)}-
				w_{(1,0,0)}+\imath w_{(1,0,1)}); \nonumber \\
f((*,*,0)) &=& \frac{1}{4} (w_{(0,0,0)}+w_{(0,0,1)}+w_{(0,0,2)}); \nonumber\\
f((*,*,1)) &=& \frac{1}{4} (w_{(0,0,0)}+e^{2\pi\imath/3} w_{(0,0,1)}
		+ e^{4\pi\imath/3} w_{(0,0,2)}); \nonumber\\
f((0,0,*)) &=& \frac{1}{4} (w_{(0,0,0)}+ w_{(0,1,0)}+
				w_{(0,2,0)}+ w_{(0,3,0)}
	\nonumber\\
	&&+
				w_{(1,0,0)}+ w_{(1,1,0)}+
				w_{(1,2,0)}+ w_{(1,3,0)}); \nonumber \\
f((0,1,1)) &=& \frac{1}{4} (
w_{(0,0,0)} + e^{2\pi\imath/3} w_{(0,0,1)}+ e^{4\pi\imath/3} w_{(0,0,2)} 
	\nonumber\\
 && + \imath w_{(0,1,0)} + \imath e^{2\pi\imath/3} 
	w_{(0,1,1)}+\imath e^{4\pi\imath/3} w_{(0,1,2)} \nonumber\\
&& -w_{(0,2,0)}-e^{2\pi\imath/3}w_{(0,2,1)}-e^{4\pi\imath/3}w_{(0,2,2)}
	\nonumber\\
&&-\imath w_{(0,3,0)} -\imath e^{2\pi\imath/3} w_{(0,3,1)} 
	-\imath e^{4\pi\imath/3} w_{(0,3,2)} \nonumber\\
&&+ w_{(1,0,0)} + e^{2\pi\imath/3} w_{(1,0,1)}+ e^{4\pi\imath/3} w_{(1,0,2)}
	\nonumber\\
&& + \imath w_{(1,1,0)} + \imath e^{2\pi\imath/3} w_{(1,1,1)}+
	\imath e^{4\pi\imath/3} w_{(1,1,2)} \nonumber\\
&& -w_{(1,2,0)}-e^{2\pi\imath/3}w_{(1,2,1)}-e^{4\pi\imath/3}w_{(1,2,2)}
		\nonumber\\
&& -\imath w_{(1,3,0)} -\imath e^{2\pi\imath/3} w_{(1,3,1)}
	-\imath e^{4\pi\imath/3} w_{(1,3,2)}
).
\end{eqnarray}
\end{example}
Note that the difference between the sums for $f((0,*,*))$ and $f((1,*,*))$
is that the Walsh coefficients are subtracted rather than added.  In
general, when we have a fixed character $p$ in the j-th position in the schema,
the corresponding sum in the transform has a phase of $e^{2 \pi p/k_j}$.

It is straightforward to make a general theorem from these observations.
\begin{theorem}
Consider a schema with $m$ fixed characters $p_i$ at positions $j_i$.
Then the average of $f$ over that schema is
\begin{eqnarray}
\prod_q{k_q^{-1/2}}
\sum_{l_1} \sum_{l_2} \ldots \sum_{l_m} & &
e^{2 \pi \imath (l_1 p_1/k_{j_1}+l_2 p_2/k_{j_2} 
	+\ldots+l_m p_m/k_{j_m})}\nonumber\\
&& w_{(0,0,\ldots,0,l_1,0,\ldots,0,l_2,0,\ldots,0,l_m,0,\ldots)}.
			\label{schematowalsh}
\end{eqnarray}
\end{theorem}

\begin{proof}
Recall that when we write $f$ in terms of its generalized Walsh coefficients, the
result has $n$ sums, where $n$ is the length of the strings.  There is
a sum for each of the $n$ characters in the string of the argument.  Now,
when we take the average of $f$ over a schema which has a $*$ in the j-th
position, only the terms that have no phase change as we traverse the j-th
dimension survive; this means that only the generalized Walsh coefficients with a $0$
in the j-th position survive in the final result.
\end{proof}

We can also define schemata in the space of real numbers,
and take schema averages using generalized Walsh coefficients as before.  
These schemata are analogous to the slices used in Goldberg~(1990a) rather
than the schemata for real parameters defined in Wright~(1991), and are
a natural extension of schemata in finite alphabets.

\begin{example}
Consider a function $f$ over $(\infty,\infty,3,3,3)$-ary strings.
Let
(0.67,*,*,*,1) refer to the set of strings whose first variable $x_1$
equals 0.67 and last variable $x_5$ equals 1.  Then the average of the
function $f$ over that schema is
\begin{equation}
f((0.67,*,*,*,1))=3^{-3/2}\sum_{j_1=-\infty}^{\infty} {\sum_{j_5=0}^2{
        w_{(j_1,0,0,0,j_5)} e^{2 \pi \imath (0.67 j_1 + 1 j_5/3)}
}}.
\end{equation}
\end{example}
To get more familiar with using these generalized Walsh coefficients on
real variables, let us do some more schema averages:
\begin{eqnarray}
f((*,*,*,*,*)) &=& 3^{-3/2} w_{(0,0,0,0,0)}; \nonumber\\
f((*,*,*,*,2)) &=& 3^{-3/2} (w_{(0,0,0,0,0)} +
        e^{2 \pi \imath/3} w_{(0,0,0,0,1)} +
        e^{4 \pi \imath/3} w_{(0,0,0,0,2)}); \nonumber\\
f((0,*,*,*,*)) &=& 3^{-3/2} \sum_{j_1=-\infty}^{\infty}{w_{(j_1,0,0,0,0)}}.
\end{eqnarray}

This chapter has generalized the notion of schema used in analyzing
genetic algorithms of binary strings and showed how to use generalized
Walsh functions and transforms to compute schema averages.  A method of
comparing schema averages to determine deception is given in the
next chapter.



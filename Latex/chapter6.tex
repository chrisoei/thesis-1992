\typeout{}
\typeout{Compiling chapter6.tex}

\chapter{Variance of Fitness}
Although the schema theorem predicts the expected growth rate of a schema
in a population, the finite population size introduces a sampling error
that can lead to large deviations from expectations and can affect GA
convergence 
(Goldberg \& Rudnick, 1991; 
Goldberg, Deb, \& Clark, 1991; 
Rudnick \& Goldberg, 1991; 
Schaffer, Eschelman, \& Offutt, 1991).

The full theory of how this sampling error affects GA behavior is beyond
the scope of this thesis.  Only the methods for calculating the signal
and noise in a competition partition (Rudnick \& Goldberg, 1991) will be
extended to non-binary alphabets.

\begin{definition}[Index Set]
The index set $G(h)$ of a schema $h$ is the set of indices of the
generalized Walsh coefficients used to express the function average
over the schema $h$.
\end{definition}
\begin{example}
In the space of strings of length 3 taken from a ternary alphabet,
\begin{eqnarray}
G((*,*,*))&=&\{(0,0,0)\},
	\nonumber\\
G((0,*,*))&=&G((1,*,*))=G((2,*,*))=\{(0,0,0),(1,0,0),(2,0,0)\}=(*,0,0),
	\nonumber\\
G((0,0,*))&=&G((1,2,*))=(*,*,0).
\end{eqnarray}
This notation helps to make the definitions of signal and noise clearer
and more compact.

\end{example}

\begin{definition}[Collateral Noise]
The collateral noise $\sigma_h$ for a schema $h$ is defined by the
following:
\begin{eqnarray}
\sigma_h^2=\mbox{\rm var}(f(h))&=&\frac{1}{|h|} \sum{\left|f(x)-\frac{1}{|h|}
	\sum{f(y)}\right|^2};\nonumber\\
&=& \frac{1}{|h|} \sum{|f(x)|^2} -\left|\frac{1}{|h|} \sum{f(x)}\right|^2;
	\nonumber\\
&=& \dirac{f(x)^2}_h-\dirac{f(x)}_h^2.
	\label{collat}
\end{eqnarray}
$\dirac{f(x)}_h$ indicates the average of $f(x)$ where
$x$ ranges over schema $h$, $\dirac{f^2(x)}_h$ indicates the
average of $f^2(x)$ over $h$,
and $|h|$ is the number of individuals represented by schema $h$.
\end{definition}
This definition differs from Rudnick and Goldberg~(1991), which defines
the collateral noise as $\sigma_h^2$.

\begin{definition}[Partition Signal]
The signal $S(J)$ of a competition partition $J$ is defined as
\begin{equation}
S^2(J)=\dirac{|f(h)|^2}_J-|\dirac{f(h)}_J|^2.
	\label{signal}
\end{equation}
\end{definition}
$\dirac{f(h)}_J$ denotes the average of schema averages $f(h)$ where
$h$ runs over $J$.  $\dirac{|f(h)|^2}_J$ denotes the average of the
absolute value squares of schema averages in the competition partition $J$.

The first term in Equation~\ref{signal} can be written as
\begin{equation}
\dirac{|f(h)|^2}_J=\frac{1}{|J|}\sum_{h \epsilon J}{|f(h)|^2},
\end{equation}
where $|J|$ is the number of schemata in J.  Substituting the
expression for $f(h)$ in terms of generalized Walsh coefficients gives
\begin{equation}
\dirac{|f(h)|^2}_J=\frac{1}{|J|} \prod_m{k_m^{-1}}
	\sum_{h \epsilon J}
	\left| \sum_{\vec\jmath \epsilon G(h)}
		{w_{\vec\jmath}
		\Psi^{(\vec k)}_{\vec\jmath}(h)} \right|^2.
\end{equation}
Expanding the quadratic yields
\begin{equation}
\dirac{|f(h)|^2}_J=\frac{1}{|J|} \prod_m{k_m^{-1}}
	\sum_{h \epsilon J}
	\sum_{\vec\jmath,\vec l \epsilon G(h)}{
		w_{\vec\jmath}\, \overline{w}_{\vec l}
		\Psi_{\vec\jmath \ominus \vec l}^{(\vec k)}(h)},
\end{equation}
where $j \ominus l=(j_1-l_1 \mbox{\ mod\ } k_1,j_2-l_2 \mbox{\ mod\ } k_2,
\ldots,
        j_n-l_n \mbox{\ mod\ } k_n)$.
Due to the orthogonality of the generalized Walsh functions, this reduces to
\begin{equation}
\dirac{|f(h)|^2}_J=\frac{1}{|J|} \prod_m{k_m^{-1}}
	\sum_{\vec\jmath \epsilon G(h)}{|w_{\vec\jmath}|^2}.
\end{equation}

The second term in Equation~\ref{signal} reduces to the following product:
\begin{equation}
|\dirac{f(h)}_J|^2=\prod_m{k_m^{-1}} |w_0|^2.
\end{equation}
Thus, Equation~\ref{signal} simplifies to:
\begin{eqnarray}
S^2(J)&=&\sum_{\vec\jmath\epsilon G(J)} (|w_{\vec\jmath}|^2-|w_0|^2);
	\nonumber\\
	&=&\sum_{\vec\jmath\epsilon G(J)-\{0\}} |w_{\vec\jmath}|^2.
		\label{signalwalsh}
\end{eqnarray}

\begin{example}
Consider (3,3,3)-ary strings and the competition partition $J=(F,F,*)$.
The square of the partition signal of J is
\begin{eqnarray}
S^2(J)&=& |w_{(0,1,0)}|^2 + |w_{(0,2,0)}|^2\nonumber\\
	&&+ |w_{(1,0,0)}|^2 + |w_{(1,1,0)}|^2+ |w_{(1,2,0)}|^2
		\nonumber\\
	&&+ |w_{(2,0,0)}|^2 + |w_{(2,1,0)}|^2+ |w_{(2,2,0)}|^2.
\end{eqnarray}
\end{example}
\begin{definition}[Partition RMS Noise]
The root-mean-squared noise $C(J)$ of a competition partition $J$ is
defined as the
root-mean-square of the collateral noises for each of the
schema $h$ in the partition.
\begin{equation}
C^2(J)=\dirac{\sigma_h^2}_J
\end{equation}
\end{definition}
In Chapter~3, there were some examples (\ref{varone},\ref{vartwo})
of computing the $\dirac{f(x)^2}_h$ term in the expression for
$\sigma_h^2$ (\ref{collat}).  These examples can be generalized
to give the following result.
For the complete derivations for binary strings, see
Goldberg \& Rudnick~(1991) and Rudnick \& Goldberg~(1991).
\begin{equation}
\sigma_h^2=\prod_m{k_m^{-1}}\sum_{(\vec\jmath,\vec l)\epsilon G^2_{\ominus}(h)-G^2(h)}{
\overline{w}_{\vec\jmath}\, w_{\vec l}
\Psi^{(\vec k)}_{\vec l \ominus \vec\jmath}(h)}
\end{equation}
$G^2(h)=G(h)\times G(h)$, $G^2_{\ominus}(h)=\{(\vec\jmath,\vec l):\vec \jmath
	\ominus \vec l \epsilon G(h)\}$, and $\vec\jmath\ominus\vec l=
	(j_1-l_2 \mbox{\ mod\ }k_1,j_2-l_2 \mbox{\ mod\ }k_2,\ldots,
		j_n-l_n \mbox{\ mod\ }k_n)$.
As before, the off-diagonal terms vanish, leaving
\begin{equation}
C^2(J)=\sum_{\vec\jmath\epsilon\tilde{G}(J)}{|w_{\vec\jmath}|^2}
\end{equation}
where $\tilde{G}(J)$ is the complement of the set $G(J)$.

\begin{example}
Consider again the above example using
(3,3,3)-ary strings and the competition partition $J=(F,F,*)$.
The square of the partition RMS noise of J is
\begin{eqnarray}
C^2(J)&=& |w_{(0,0,1)}|^2 + |w_{(0,0,2)}|^2 + |w_{(0,1,1)}|^2+|w_{(0,1,2)}|^2\nonumber\\
        &&+ |w_{(0,2,1)}|^2 + |w_{(0,2,2)}|^2+ |w_{(1,0,1)}|^2+|w_{(1,0,2)}|^2
                \nonumber\\
        &&+ |w_{(1,1,1)}|^2 + |w_{(1,1,2)}|^2+ |w_{(1,2,1)}|^2+|w_{(1,2,2)}|^2
                \nonumber\\
        &&+ |w_{(2,0,1)}|^2 + |w_{(2,0,2)}|^2+ |w_{(2,1,1)}|^2+|w_{(2,1,2)}|^2
                \nonumber\\
        &&+ |w_{(2,2,1)}|^2 + |w_{(2,2,2)}|^2.
\end{eqnarray}
\end{example}


For large alphabets
and fairly smooth fitness functions, we can use the alphabet-size reduction
described earlier to get a good approximation to the variances with
relatively little effort.  For instance, the variance of Test Function 2 is
\begin{eqnarray}
\int_0^1{(x^2 (1-x)^2 (1/2-x)^3)^2 dx}&&\nonumber\\
	-\left(\int_0^1{(x^2(1-x)^2(1/2-x)^3) dx}
	\right)^2&=&1.734\times 10^{-7}.
\end{eqnarray}
We can approximate the same calculation using the 7-ary reduction.
\begin{equation}
\frac{1}{7}(|w_1|^2+|w_2|^2+|w_3|^2+|w_4|^2+|w_5|^2+|w_6|^2)=1.723\times 10^{-7}.
\end{equation}

This chapter has shown that the signal and noise calculations for functions
over binary strings can be generalized to non-binary strings as well.
The signal-to-noise calculations and the Hadamard transforms 
are tools which examine two independent
modes by which a genetic algorithm can fail to converge to the global optimum.



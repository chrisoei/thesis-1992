\typeout{}
\typeout{Compiling chapter1.tex}
\chapter{Introduction}
\hyphenation{Bledsoe Kauffman}
\section{Preview}
Walsh functions, introduced by the mathematician J.~L.~Walsh in 1923, 
recently have become popular tools in fields such as 
telecommunications engineering, radar systems, image recognition and
processing, speech processing, coding systems, and spectroscopy
(Beauchamp, 1975).
Bethke~(1981) introduced the Walsh-schema transform, a method of using 
Walsh functions for
computing schema averages.  Since then, the Walsh-schema transform has become
a cornerstone in the theory of genetic algorithms (GAs).

Most of the theory of genetic algorithms applies to GAs over binary strings,
real numbers, or permutations.  
Attempts to generalize this theory borrowed some notions from set theory
(Radcliffe, 1991; Vose \& Liepins, 1991).
Recently, Mason~(1991) extended the concept of
partition coefficients (Bethke, 1981) from the theory of binary-coded
genetic algorithms to GAs over non-binary strings.

This thesis generalizes Walsh functions to non-binary strings and
reworks some of the existing theory of GAs to incorporate these new
functions.  Topics that will be covered and extended include Bethke's Walsh
schema transform, real-coded GAs (Bledsoe, 1961; Goldberg, 1990a;
Wright, 1991), detecting static deception and \linebreak Hadamard transforms 
(Homaifar \& Qi, 1990; Homaifar, Qi, \& Fost, 1991), and the variance of 
fitness (Goldberg, Deb, \& Clark, 1991; Goldberg \& Rudnick, 1991;
Rudnick \& Goldberg, 1991).

\section{Notation}
Throughout this thesis, $n$ will refer to the length of the strings in
the domain of the fitness function.
$k$ will refer to the cardinality of the alphabet, or if each
character of a string is taken from a different alphabet, $k_1$ refers
to the alphabet for the first character, $k_2$ for the second, and so on.
A k-ary alphabet is an alphabet with k characters.
The characters in a k-ary alphabet will be represented by the
integers 0 through $k-1$.
A k-ary string is a string all of whose characters are taken from 
the same k-ary alphabet.
A string that belongs to a \veckary alphabet is a string whose first
character belongs to a $k_1$-ary alphabet, whose second character 
belongs to a $k_2$-ary alphabet, etc; such strings will be referred to 
as \veckary strings.
A string will be represented by a vector whose components are the
integer representations of the characters in the string.
For example, the hexidecimal (16-ary) string ``3F2'' would be denoted by (3,15,2).
k-ary Walsh functions are the generalization of Walsh functions to k-ary
alphabets, and \veckary Walsh functions are the generalization to
strings that belong to \veckary alphabets.  Both of these generalizations
of Walsh functions, as well as generalizations of the Walsh functions to
strings with characters which are real numbers, will be referred to as 
generalized Walsh functions.  The corresponding transforms will be
referred to as k-ary and \veckary Walsh transforms.


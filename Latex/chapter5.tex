\typeout{}
\typeout{Compiling chapter5.tex}

\chapter{Using Generalized Hadamard Transforms to Detect Deception}
\section{Deception in Non-binary Strings}
Hadamard transforms provide a convenient way of checking
deceptive conditions in a systematic manner (Homaifar, Qi, Fost, 1991;
Homaifar, Qi, 1990).  In this chapter, the Hadamard transform
will be generalized to non-binary strings.

Deception in functions over non-binary strings is qualitatively different than
deception in functions over binary strings.  In order to get full deception
with binary strings, all schemas of order $n-1$ or less must point towards
the complement of the global optimum.  With non-binary strings, all that
is required is that all schemas of order $n-1$ or less point away from the
global optimum.  Mason uses this as his definition of deception (Mason, 1991).

\begin{example}
Consider a function $f$ over a ternary string of length
3 whose global optimum is at (2,2,2).  Then some of the conditions necessary
for deception are as follows:
\begin{eqnarray}
f((2,*,*)) &<& f((0,*,*)) \mbox{\ or\ } f((1,*,*)); \nonumber\\
f((2,2,*)) &<& f((0,0,*)) \mbox{\ or\ } f((0,1,*)) \mbox{\ or\ } f((1,0,*)) \mbox{\ or\ } f((1,1,*)); \nonumber\\
f((2,2,*)) &<& f((1,2,*)) \mbox{\ or\ } f((0,2,*)); \nonumber \\
f((2,2,*)) &<& f((2,0,*)) \mbox{\ or\ } f((2,1,*)); \nonumber \\
f((1,2,*)) &<& f((1,0,*)) \mbox{\ or\ } f((1,1,*)).
\end{eqnarray}
\end{example}

Mutation for non-binary strings works differently than mutation for binary
strings.  Some mutation operators, such as creeping mutation, mutate
characters to nearby characters.  For example, using a 10-ary alphabet,
the string (0,0,0) is much more likely to be perturbed to (1,0,0) than (9,0,0).
Similarly, mutations on a real number might be implemented by adding noise
with a gaussian distribution.  If the fitness function is not too
discontinuous, then these kinds of mutations can work as a sort of gradient
descent to help convergence (Bledsoe, 1961).  The analysis of how the distribution of
perturbations affects convergence is beyond the scope of this thesis
and overlaps the theory of simulated annealing (Kirkpatrick,
Gelatt, \& Vecci, 1983; Szu \& Hartley, 1987).
Assuming that the algorithm uses localized perturbations, it is reasonable to define a fully
deceptive function as a function whose order $n-1$ or less schemas point as far
away from the global optima as possible.  Now the deceptive conditions
become more straightforward:
\begin{eqnarray}
f((0,*,*)) &>& f((1,*,*)), \nonumber\\
f((0,*,*)) &>& f((2,*,*)), \nonumber\\
f((0,0,*)) &>& f((0,1,*)), \nonumber\\ 
f((0,0,*)) &>& f((0,2,*)), \nonumber\\ 
f((0,0,*)) &>& f((1,0,*)), \nonumber\\ 
f((0,0,*)) &>& f((1,1,*)), \nonumber\\ 
f((0,0,*)) &>& f((1,2,*)), \nonumber\\
f((0,0,*)) &>& f((2,0,*)), \nonumber\\
f((0,0,*)) &>& f((2,1,*)), \nonumber\\
f((0,0,*)) &>& f((2,2,*)),
\end{eqnarray} 
and all permutations of the strings above.

\section{Hadamard Transform}
Consider the competition partition in which the fixed positions
are at $j_1,j_2,\ldots,j_p$, where $p$ is the order of the partition.
\begin{definition}[Generalized Hadamard Transform]
Define the generalized Hadamard transform matrix $H$ as:
\begin{equation} 
H=h_1 \otimes h_2 \otimes \ldots \otimes h_p,
\end{equation}
where $\otimes$ is the tensor product:
\begin{equation}
\left(\begin{array}{cc}
	a & b\\
	c & d
	\end{array} \right)
\otimes \left(\begin{array}{cc}
	e & f\\
	g &h
	\end{array} \right)
= \left(\begin{array}{cccc}
	ae& af& be& bf\\
	ag& ah& bg& bh \\
	ce& cf& de& df \\
	cg& ch& dg& dh
\end{array}\right),
\end{equation}
and $h_m$ is defined as follows:
\begin{equation}
h_m = \left(\begin{array}{ccccc}
		1 & 1& 1& \ldots &1 \\
		1 & e^{2 \pi \imath /k_{j_m}} & e^{4 \pi \imath/k_{j_m}} &\ldots &
			e^{(k_{j_m}-1) 2 \pi \imath/k_{j_m}} \\
		\vdots &\vdots &\vdots &\ddots &\vdots \\
		1 & e^{(k_{j_m}-1) 2 \pi \imath/k_m} &
			e^{(k_{j_m}-1) 4 \pi \imath/k_m} &\ldots&
			e^{(k_{j_m}-1) (k_{j_m}-1) 2 \pi \imath/k_{j_m}}
	\end{array}
	\right).
\end{equation}
\end{definition}
The generalized Hadamard transform will be used in the next section to
compute the conditions for deception.
\section{Detecting Deception}
Now we make the important assumption that the global optimum is at
$(k_1-1,k_2-1,\ldots,k_n-1)$ and all schemas of order $n-1$ or less
must point to $(0,0,\ldots,0)$.  
This definition of deception requires that the lower-order schemata
lead as far away from the global optimum as possible.
This differs from Mason's definition of deception, which requires only
that the lower-order schemata do not point to the global optimum.
Using this new definition, we can now define the
matrix $M$ such that the deceptive conditions for that partition become
$M W>0$, where
\begin{equation}
M= \left(\begin{array}{c}
	\mbox{1st row of H-2nd row of H} \\
	\mbox{1st row of H-3rd row of H} \\
	\vdots \\
	\mbox{1st row of H-last row of H}
	\end{array}
	\right),
\end{equation}
and $W$ is the vector of generalized Walsh coefficients used in the
competition partition.

\begin{example}
Let $w$ be the Walsh coefficients for the fitness function $f$
over (3,2,2)-ary strings.  Consider the competition partition $(F,*,*)$, where
$F$ represents a fixed position and * represents a wildcard character.  Then
\begin{eqnarray}
W&=&(w_{(0,0,0)},w_{(1,0,0)},w_{(2,0,0)})^T, \nonumber\\
H&=&\left(\begin{array}{ccc}
	1&1&1\\
	1&e^{2 \pi \imath/3}&e^{4 \pi \imath/3}\\
	1&e^{4 \pi \imath/3}&e^{8 \pi \imath/3}
	\end{array}
	\right), \nonumber\\
M&=&\left(\begin{array}{ccc}
	0&1-e^{2 \pi \imath/3}&1-e^{4 \pi \imath/3} \\
	0&1-e^{4 \pi \imath/3}&1-e^{8 \pi \imath/3}
	\end{array}
	\right).
\end{eqnarray}
The condition that $MW>0$ becomes the following:
\begin{eqnarray}
(1-e^{2\pi \imath/3})w_{(1,0,0)}+(1-e^{4\pi\imath/3})w_{(2,0,0)} &>&0;\nonumber\\
(1-e^{4\pi \imath/3})w_{(1,0,0)}+(1-e^{8\pi\imath/3})w_{(2,0,0)} &>&0.
\end{eqnarray}
When we substitute function values for the Walsh coefficients in the above
expression, we get
\begin{eqnarray}
f((0,*,*)) &>& f((1,*,*)), \nonumber\\
f((0,*,*)) &>& f((2,*,*)),
\end{eqnarray}
which are indeed the deceptive conditions corresponding to that partition.
\end{example}

\begin{example}
Let us repeat the above calculations for the competition partition
$(F,D,F)$:
\begin{eqnarray}
W&=&(w_{(0,0,0)},w_{(0,0,1)},w_{(1,0,0)},w_{(1,0,1)},w_{(2,0,0)},w_{(2,0,1)})^T,
	\nonumber\\
H&=&\left(\begin{array}{cccccc}
	1&1&1&1&1&1\\
	1&-1&1&-1&1&-1\\
	1&1&e^{2 \pi\imath/3}&e^{2\pi\imath/3}&e^{4\pi\imath/3}&
		e^{4\pi\imath/3}\\
	1&-1&e^{2\pi\imath/3}&-e^{2\pi\imath/3}&e^{4\pi\imath/3}&
		-e^{4\pi\imath/3}\\
	1&1&e^{4\pi\imath/3}&e^{4\pi\imath/3}&e^{8 \pi\imath/3}&
		e^{8\pi\imath/3}\\
	1&-1&e^{4\pi\imath/3}&-e^{4\pi\imath/3}&e^{8 \pi\imath/3}&
		-e^{8\pi\imath/3}
	\end{array} \right), \nonumber\\
M&=&\left(\begin{array}{cccccc}
	0&2&0&2&0&2\\
	0&0&1-A &1-A&1-B&1-B\\
	0&2&1-A&1+A&1-B&1+B\\
	0&0& 1-B&1-B &1-A&1-A\\
	0&2&1-B&1+A&1-A&1+B
\end{array} \right),
\end{eqnarray}
where $A=e^{2 \pi \imath/3}$ and $B=e^{-2 \pi \imath/3}$.  $MW>0$ gives
\begin{eqnarray}
2 w_{(0,0,1)} + 2 w_{(1,0,1)} + 2 w_{(2,0,1)} &>& 0,\nonumber\\
(1-A) w_{(1,0,0)} + (1-A) w_{(1,0,1)} + (1-B) w_{(2,0,0)}+(1-B) w_{(2,0,1)}&>&0,
	\nonumber\\
2 w_{(0,0,1)} + (1-A) w_{(1,0,0)} +(1+A) w_{(1,0,1)} + (1-B) w_{(2,0,0)} +&&
	\nonumber\\
	(1+B) w_{(2,0,1)} &>& 0, \nonumber\\
(1-B) w_{(1,0,0)} +(1-B)w_{(1,0,1)} + (1-A)w_{(2,0,0)}+(1-A)w_{(2,0,1)}&>&0,
	\nonumber\\
2 w_{(0,0,1)} +(1-B) w_{(1,0,0)} +(1+B) w_{(1,0,1)} +(1-A) w_{(2,0,0)}+&&
	\nonumber\\
	(1+A)w_{(2,0,1)} &>&0.
\end{eqnarray}
Translated into function values, this gives the following set of
inequalities:
\begin{eqnarray}
f((0,*,0)) &>& f((0,*,1)), \nonumber\\
f((0,*,0)) &>& f((1,*,0)), \nonumber\\
f((0,*,0)) &>& f((1,*,1)), \nonumber\\
f((0,*,0)) &>& f((2,*,0)), \nonumber\\
f((0,*,0)) &>& f((2,*,1)),
\end{eqnarray}
which are the deceptive conditions corresponding to the competition
partition $(F,D,F)$.
\end{example}

For a function to be deceptive at order $p$ requires that all order $p$
schemas lead as far away to the global optimum as possible.  Since there
are $n \choose p$ ways of choosing $p$ fixed positions
among the $n$ characters, there are that many competition partitions at
that order.  In the above example, the function is deceptive at order 1 if
it is deceptive in the partitions (F,D,D), (D,F,D), and (D,D,F).
For full deception, the function must be deceptive at all orders between
1 and $n-1$ inclusive.  Mathematica routines for computing all of the
above are included in Appendix~A.

This chapter has defined the deceptive conditions for functions over
non-binary alphabets.  As mentioned in Chapter~2, deception is only one
of the many reasons why a genetic algorithm may fail.  The next chapter
focusses on another mode of failure: the variances of schema fitness.



\typeout{}
\typeout{Compiling chapter7.tex}

\chapter{Conclusion}
This thesis has shown how the Walsh functions and many of the techniques
that use it can be generalized to non-binary alphabets in a natural 
and straightforward way.  This conclusion focuses on possible topics
for future research that use generalized Walsh functions.

Directions for future research using generalized Walsh functions and
transforms include generalizing more of the theory for binary strings,
such as the nonuniform Walsh-schema transform
(Bridges \& Goldberg, 1991), creating deceptive problems
(Liepins \& Vose, 1990b; Liepins \& Vose, 1991; Whitley, 1991a),
operator-adjusted Walsh coefficients (Goldberg, 1989c),
and the sufficient conditions for deception (Deb, Goldberg \& 1992).

Another possibility is a probabilistic approach towards deception.
Checking for deception requires evaluating the fitness function over
every single point in the entire space, and for this reason, it has never
been done for anything but tractable problems or problems that were
handmade to be difficult.  Given the empirical evidence that smoother
functions are easier to optimize, it seems possible that the probability
that a function is deceptive depends on the asymptotic behavior
of the generalized Walsh coefficients.  It would be useful to have either
analytical results or a table of probabilities that relate how quickly
the coefficients decay, the length of the string, the cardinality,
and the order of static deception.  This approach might give a way
of determining which representation is the likeliest to work for a given
problem, and whether a GA is likely to solve the problem or not,
using only knowledge about the smoothness of the fitness function.

In the theory of nonlinear systems, there are two methods for finding
solutions.  The first is to make a linear approximation; the second
is to use some symmetry of the system to reduce the dimensionality
of the problem.  The current theory for genetic algorithms is largely
based on the first method; it makes predictions based on a linearized
model of the system at generation zero.  As Grefenstette~(1991)
pointed out, the validity of this linear approximation as time
progresses is open to
question.  Methods of analysis based on a symmetry of the system would
be valid for any length of time.  N. Packard~(personal communication,
1991) and this author have
speculated that a renormalization group technique such as that used
in analyzing lattice processes in physics could also be used for analyzing
GAs.  The genetic algorithm, not including the user-defined fitness
function, appears to have no fundamental length scale other than
the string length, which corresponds to the lattice size in physics.
Also, the way that low-order schemata combine to form higher-order
schemata is reminiscent of a phase transition.  These two facts, along
with some numerical experiments, suggest that scaling does occur in
some situations and a renormalization group approach would be
promising.  A renormalization group approach would involve treating
blocks of characters in the string as a single character of higher cardinality,
and then asking what would a genetic algorithm with this new representation do.
This thesis, which provides a framework for a theory of genetic
algorithms over non-binary strings, is a step in that direction.


